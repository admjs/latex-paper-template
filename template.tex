% ------------------------------------------------
% LaTeX Template for ML/AI Research Paper
% ------------------------------------------------

\documentclass[11pt]{article}

% ------------------------------------------------
% Packages
% ------------------------------------------------
\usepackage{amsmath, amssymb}
\usepackage{graphicx}
\usepackage{hyperref}
\usepackage{natbib}          % For bibliography
\usepackage{geometry}        % For adjusting margins
\usepackage{times}           % For Times New Roman font
\usepackage{float}           % For controlling float positions
\usepackage{booktabs}        % For professional-looking tables
\usepackage{caption}
\usepackage{subcaption}
\usepackage{color}

% ------------------------------------------------
% Page Setup
% ------------------------------------------------
\geometry{margin=1in}

% ------------------------------------------------
% Title and Author Information
% ------------------------------------------------
\title{Your Paper Title Here}
\author{
  Your Name\thanks{Corresponding author: \texttt{your.email@example.com}} \\
  \normalsize Your Affiliation \\
  \and
  Co-author Name \\
  \normalsize Co-author Affiliation \\
  % Add more authors if necessary
}
\date{\today}

% ------------------------------------------------
% Document Begins
% ------------------------------------------------
\begin{document}

\maketitle

% ------------------------------------------------
% Abstract
% ------------------------------------------------
\begin{abstract}
% Provide a concise summary of the entire paper (150-250 words).
% Include background, methods, results, and conclusions.
\end{abstract}

% Optionally, you can add keywords
% \vspace{0.5cm}
% \noindent\textbf{Keywords:} Keyword1, Keyword2, Keyword3

% ------------------------------------------------
% Main Content
% ------------------------------------------------

% 1. Introduction
\section{Introduction}
% Introduce the topic, establish the importance, and state the main contributions.
% - Context
% - Problem Statement
% - Objectives
% - Contributions
% - Structure of the paper

% 2. Methodology
\section{Methodology}
% Provide a detailed description of your approach or model.
% - Theoretical Background
% - Model Architecture
% - Algorithms
% - Assumptions
% - Implementation Details

% 3. Experiments
\section{Experiments}
% Describe how you tested your model and ensure reproducibility.
% - Datasets
% - Experimental Setup
% - Hyperparameters
% - Evaluation Metrics

% 4. Results
\section{Results}
% Present the findings of your experiments.
% - Quantitative Results (Use tables and figures)
% - Qualitative Results
% - Comparisons
% - Statistical Analysis

% Example of including a figure
% \begin{figure}[H]
%     \centering
%     \includegraphics[width=0.7\textwidth]{path/to/your/image.png}
%     \caption{Caption of the figure.}
%     \label{fig:label}
% \end{figure}

% Example of including a table
% \begin{table}[H]
%     \centering
%     \caption{Caption of the table.}
%     \label{tab:label}
%     \begin{tabular}{lcc}
%         \toprule
%         Header1 & Header2 & Header3 \\
%         \midrule
%         Row1 & Data & Data \\
%         Row2 & Data & Data \\
%         \bottomrule
%     \end{tabular}
% \end{table}

% 5. Discussion
\section{Discussion}
% Interpret the results and discuss their implications.
% - Insights
% - Limitations
% - Practical Implications
% - Theoretical Implications

% 6. Conclusion
\section{Conclusion}
% Summarize the research and suggest future directions.
% - Summary of findings
% - Significance
% - Future Work

% 7. Related Work
\section{Related Work}
% Situate your research within the existing body of work.
% - Literature Review
% - Critical Analysis
% - Positioning

% 8. Acknowledgments
\section*{Acknowledgments}
% Credit those who contributed to the research but are not listed as authors.
% - Funding Sources
% - Collaborations

% ------------------------------------------------
% References
% ------------------------------------------------
\bibliographystyle{plainnat}
\bibliography{references}

% ------------------------------------------------
% Appendices (Optional)
% ------------------------------------------------
\appendix

\section{Appendix Title}
% Include supplementary material that supports the paper but is too detailed for the main text.
% - Mathematical Proofs
% - Additional Figures or Tables
% - Code Listings

% Example of an equation
% \begin{equation}
%     E = mc^2
% \end{equation}

% ------------------------------------------------
% Document Ends
% ------------------------------------------------
\end{document}
